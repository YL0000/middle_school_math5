\chapter{向量几何初步}
在第三章,我们学习了向量运算与运算律,这一章我们
要用向量代数的方法来研究几何学,把对几何学的研究推进
到有效能算的定量的水平。

\section{平行与相似}
\subsection{直线的向量方程}
\begin{figure}[htp]
    \centering
\begin{tikzpicture}[>=latex, yscale=1.5]
    \tkzDefPoints{2/.5/A, .5/2/P, 0/0/O}
    \tkzDefMidPoint(A,P)
    \tkzGetPoint{B}
\draw[<->,  thick](P)--(O)--(A);
\draw[->,  thick](O)--(B);
\tkzDrawLines[add =.25 and .25](A,P)
\tkzLabelPoints[right](A,B,P)
\tkzLabelPoint[below](O){$O$}
\end{tikzpicture}
    \caption{}
\end{figure}

给定空间任意两点$A$、$B$(图4.1), 由平行向量基本定理可知,对空
间中一点$P$与$A$、$B$两点共线的充要条件是存在一实数$t$, 使
\[\Vec{AP}=t\Vec{AB}\]
对空间任一点$O$, 这个条件还可
写为
\[\Vec{OP}-\Vec{OA}=t\left(\Vec{OB}-\Vec{OA}\right)\]
\begin{equation}
    \Vec{OP}=(1-t)\Vec{OA}+t\Vec{OB}
\end{equation}
这就是说,如果$P$点在直线$AB$上,则一定存在实数$t$使(4.1)
式成立,反之,任给一实数$t$, 由等式(4.1)所确定的$P$点也
一定在直线$AB$上,方程(4.1)通常叫做\textbf{直线$AB$的向量方
程}。其中\textbf{参数}$t$的几何意义是,$|t|=|\Vec{AP}|:|\Vec{AB}|$, 当$P$点在
射线$AB$上,$t\ge 0$. 当$P$在射线$AB$的反向延长线上,$t<0$.

由直线$AB$的向量方程(4.1)可知,如果
\[\Vec{OP}=x\Vec{OA}+y\Vec{OB}\]
那么点$P$在直线$AB$上的充要条件是$x+y=1$

\begin{example}
    已知$\vec{a},\vec{b}$是两个线性无关的向量,
$\Vec{OA}=\alpha\vec{a}$, $\Vec{OB}=\beta\vec{b}\quad (\alpha\ne 0,\;\beta\ne 0)$,$\Vec{OC}=x\vec{a}+y\vec{b}$(图4.2),
则$A$、$B$、$C$三点共线的充要条件是
\[\frac{x}{\alpha}+\frac{y}{\beta}=1\]
\end{example}

\begin{figure}[htp]
    \centering
    \begin{tikzpicture}[>=latex]
  \draw(0,0)--(-40:4);
\draw[->](0,-3)node[below]{$O$}--(-40:.5)node[right]{$B$}node[below left]{$\beta\vec{b}$};
\draw[->](0,-3)--(-40:1.5)node[right]{$C$};
\draw[->](0,-3)--(-40:2.5)node[right]{$A$}node[below]{$\alpha\vec{a}$};  

\tkzDefPoint(0,-3){O}
\tkzDefPoint(-40:.5){B}
\tkzDefPoint(-40:2.5){A}
\tkzDefMidPoint(O,B) \tkzGetPoint{B1}
\tkzDefMidPoint(O,A) \tkzGetPoint{A1}
\draw[->](O)--(A1)node[right]{$\vec{a}$};
\draw[->](O)--(B1)node[left]{$\vec{b}$};

\end{tikzpicture}
    \caption{}
\end{figure}

\begin{proof}
    由直线的向量方程可知,点$C$在直线$AB$上的充要条
    件是存在实数$t$使
\[\Vec{OC}=(1-t)\Vec{OA}+t\Vec{OB}\]
即\[\Vec{OC}=(1-t)\alpha\vec{a}+t\beta\vec{b}\]
但已知$\Vec{OC}=x\vec{a}+y\vec{b}$且$\vec{a}\nparallel\vec{b}$,所以
\[x=(1-t)\alpha,\qquad y=t\beta\]
消去$t$则可得$A$、$B$、$C$三点共线的充要条件为
\[\frac{x}{\alpha}+\frac{y}{\beta}=1\]
\end{proof}



























































