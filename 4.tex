\chapter{向量几何初步}
在第三章,我们学习了向量运算与运算律,这一章我们
要用向量代数的方法来研究几何学,把对几何学的研究推进
到有效能算的定量的水平。

\section{平行与相似}
\subsection{直线的向量方程}
\begin{figure}[htp]
    \centering
\begin{tikzpicture}[>=latex, yscale=1.5]
    \tkzDefPoints{2/.5/A, .5/2/P, 0/0/O}
    \tkzDefMidPoint(A,P)
    \tkzGetPoint{B}
\draw[<->,  thick](P)--(O)--(A);
\draw[->,  thick](O)--(B);
\tkzDrawLines[add =.25 and .25](A,P)
\tkzLabelPoints[right](A,B,P)
\tkzLabelPoint[below](O){$O$}
\end{tikzpicture}
    \caption{}
\end{figure}

给定空间任意两点$A$、$B$(图4.1), 由平行向量基本定理可知,对空
间中一点$P$与$A$、$B$两点共线的充要条件是存在一实数$t$, 使
\[\Vec{AP}=t\Vec{AB}\]
对空间任一点$O$, 这个条件还可
写为
\[\Vec{OP}-\Vec{OA}=t\left(\Vec{OB}-\Vec{OA}\right)\]
\begin{equation}
    \Vec{OP}=(1-t)\Vec{OA}+t\Vec{OB}
\end{equation}
这就是说,如果$P$点在直线$AB$上,则一定存在实数$t$使(4.1)
式成立,反之,任给一实数$t$, 由等式(4.1)所确定的$P$点也
一定在直线$AB$上,方程(4.1)通常叫做\textbf{直线$AB$的向量方
程}。其中\textbf{参数}$t$的几何意义是,$|t|=|\Vec{AP}|:|\Vec{AB}|$, 当$P$点在
射线$AB$上,$t\ge 0$. 当$P$在射线$AB$的反向延长线上,$t<0$.

由直线$AB$的向量方程(4.1)可知,如果
\[\Vec{OP}=x\Vec{OA}+y\Vec{OB}\]
那么点$P$在直线$AB$上的充要条件是$x+y=1$

\begin{example}
    已知$\vec{a},\vec{b}$是两个线性无关的向量,
$\Vec{OA}=\alpha\vec{a}$, $\Vec{OB}=\beta\vec{b}\quad (\alpha\ne 0,\;\beta\ne 0)$,$\Vec{OC}=x\vec{a}+y\vec{b}$(图4.2),
则$A$、$B$、$C$三点共线的充要条件是
\[\frac{x}{\alpha}+\frac{y}{\beta}=1\]
\end{example}

\begin{figure}[htp]
    \centering
    \begin{tikzpicture}[>=latex]
  \draw(0,0)--(-40:4);
\draw[->](0,-3)node[below]{$O$}--(-40:.5)node[right]{$B$}node[below left]{$\beta\vec{b}$};
\draw[->](0,-3)--(-40:1.5)node[right]{$C$};
\draw[->](0,-3)--(-40:2.5)node[right]{$A$}node[below]{$\alpha\vec{a}$};  

\tkzDefPoint(0,-3){O}
\tkzDefPoint(-40:.5){B}
\tkzDefPoint(-40:2.5){A}
\tkzDefMidPoint(O,B) \tkzGetPoint{B1}
\tkzDefMidPoint(O,A) \tkzGetPoint{A1}
\draw[->](O)--(A1)node[right]{$\vec{a}$};
\draw[->](O)--(B1)node[left]{$\vec{b}$};

\end{tikzpicture}
    \caption{}
\end{figure}

\begin{proof}
    由直线的向量方程可知,点$C$在直线$AB$上的充要条
    件是存在实数$t$使
\[\Vec{OC}=(1-t)\Vec{OA}+t\Vec{OB}\]
即\[\Vec{OC}=(1-t)\alpha\vec{a}+t\beta\vec{b}\]
但已知$\Vec{OC}=x\vec{a}+y\vec{b}$且$\vec{a}\nparallel\vec{b}$,所以
\[x=(1-t)\alpha,\qquad y=t\beta\]
消去$t$则可得$A$、$B$、$C$三点共线的充要条件为
\[\frac{x}{\alpha}+\frac{y}{\beta}=1\]
\end{proof}

\begin{example}
    如图4.3, 设$O$、$A$、$B$三点不共线,
$\Vec{OA}=\vec{a}$, $\Vec{OB}=\vec{b}$, $\Vec{OA_1}=\alpha_1\vec{a}$, $\Vec{OA_2}=\alpha_2\vec{a}$, $\Vec{OB_1}=\beta_1\vec{b}$, $\Vec{OB_2}=\beta_2\vec{b}$
且$\alpha_1$、$\alpha_2$、$\beta_1$、$\beta_2$都不为零,又设$A_1B_2$与$A_2B_1$交于$C$点,试以$\vec{a}$、$\vec{b}$、$\alpha_1$、$\alpha_2$、$\beta_1$、$\beta_2$, 表示$\vec{c}=\Vec{OC}$
\end{example}

\begin{figure}[htp]
    \centering
\begin{tikzpicture}[>=latex]
\tkzDefPoints{0/0/O, 1/0/A, 2/0/A_1,  4.5/0/A_2}
\tkzDefPoint(45:1){B}
\tkzDefPoint(45:2.5){B_1}\tkzDefPoint(45:4.5){B_2}
\tkzDrawSegments[->, thick](O,B_1 O,B_2  O,A_1 O,A_2 A_1,B_2 B_1,A_2)
\tkzInterLL(A_1,B_2)(B_1,A_2) \tkzGetPoint{C}
\tkzDrawSegments[->, thick](O,C)
\tkzLabelPoints[below](A,A_1,A_2)
\tkzLabelPoints[above](B,B_1,B_2)
\tkzLabelPoints[right](C)
\tkzLabelPoints[left](O)
\tkzDrawPoints(A,B)
\end{tikzpicture}
    \caption{}
\end{figure}


\begin{solution}
    设$\vec{c}=x\vec{a}+y\vec{b}$, 因为$A_1$、$C$、$B_2$三点共线,$A_2$、$C$、$B_1$三点共线,由例4.1有方程组
\[\begin{cases}
    \frac{x}{\alpha_1}+\frac{y}{\beta_2}=1\\
    \frac{x}{\alpha_2}+\frac{y}{\beta_1}=1\\
\end{cases}\]
由于$A_1B_2$与$A_2B_1$相交于$C$点,可知$\alpha_1\beta_1-\alpha_2\beta_2\ne 0$, 解这个方程组可得
\[x=\alpha_1\alpha_2\frac{\beta_2-\beta_1}{\alpha_2\beta_2-\alpha_1\beta_1},\qquad y=\beta_1\beta_2\frac{\alpha_2-\alpha_1}{\alpha_2\beta_2-\alpha_1\beta_1}\]
\end{solution}


\begin{example}
    
\end{example}

\begin{solution}
    
\end{solution}

\begin{example}
    
\end{example}



\begin{solution}
    
\end{solution}

\begin{example}
    
\end{example}

\begin{solution}
    
\end{solution}



\begin{example}
    
\end{example}

\begin{example}
    
\end{example}


\begin{solution}
    
\end{solution}

\begin{solution}
    
\end{solution}

\begin{solution}
    
\end{solution}

\begin{solution}
    
\end{solution}


























































































\subsection*{习题 4.3}
\begin{enumerate}
    \item 试证圆的相交弦定理。
    \item 从圆$O$外一点$P$引圆的切线$PT_1$和$PT_2$, $T_1$、$T_2$为切点。
再引圆$O$的割线$PQR$, 交圆$O$于$Q$、$R$, 交$T_1T_2$于$T$, 
设$|PQ|=a$, $|PR|=b$, $|PT|=t$, 求证:
\[\frac{1}{a}+\frac{1}{b}=\frac{2}{t}\]
\end{enumerate}

\section*{复习题四}
\begin{enumerate}

\item 在空间中,设有线性关系$\lambda_1\vec{a}_1+\lambda_2\vec{a}_2+\lambda_3\vec{a}_3+\lambda_4\vec{a}_4+\lambda_5\vec{a}_5=\vec{0}$,且$\lambda_1\lambda_2\lambda_3\lambda_4\lambda_5\ne 0$, 若
\begin{enumerate}
    \item $\vec{a}_1,\vec{a}_2,\vec{a}_3,\vec{a}_4,\vec{a}_5$都是非零向量;
    \item 有且只有$\vec{a}_4=\vec{a}_5=\vec{0}$;
    \item 有且只有三个零向量。
\end{enumerate}
问在各种情况下,它们的几何意义
分别是什么?
\item 试作一给定有向线段$\Vec{AB}$的定比分点,其比值分别为:
$\frac{1}{2},2,-2,-\frac{1}{2}$。
\item 设$P$、$A$、$B$是共线的相异三点,
$\Vec{AP}=\rho\Vec{PB}$, 
试用$\rho$去表达下列五个实数$\alpha$、$\beta$、$\gamma$、$\delta$、$\varepsilon$.
\[\begin{split}
    \Vec{BP}&=\alpha\Vec{PA},\qquad \Vec{PA}=\beta\Vec{AB},\qquad \Vec{BA}=\gamma\Vec{AP}\\
    \Vec{PB}&=\delta\Vec{BA},\qquad \Vec{AB}=\varepsilon\Vec{BP}
\end{split}\]
\item 如图,$\ell_1,\ell_2$交于$O$点,$\vec{u},\vec{v}$是$\ell_1,\ell_2$方向的单位向量,
设
\[\begin{split}
    \Vec{OA_1}=\alpha_1\vec{u},\qquad  \Vec{OB_1}=\beta_1\vec{u},\qquad  \Vec{OC_1}=\gamma_1\vec{u}\\
    \Vec{OA_2}=\alpha_2\vec{v},\qquad  \Vec{OB_2}=\beta_2\vec{v},\qquad  \Vec{OC_2}=\gamma_2\vec{v}
\end{split}\]
试用$\vec{u},\vec{v}$的线性组合表示
$\Vec{OP}$、$\Vec{OQ}$、$\Vec{OR}$. 并证明$P$、$Q$、$R$三点共线。

\begin{figure}[htp]
    \centering
\begin{tikzpicture}[>=latex]
\tkzDefPoints{0/0/O, 2/0/A_2, 3.5/0/B_2, 5/0/C_2}
\tkzDefPoint(40:2.5){A_1}\tkzDefPoint(40:4){B_1}
\tkzDefPoint(40:5.5){C_1}
\tkzLabelPoints[above](A_1,B_1,C_1)
\tkzLabelPoints[below](A_2,B_2,C_2)
\draw(0,0)node[left]{$O$}--(6,0)node[right]{$\ell_2$};
\draw(0,0)--(40:6)node[right]{$\ell_1$};
\tkzDrawSegments(A_1,C_2 A_1,B_2 B_1,A_2 B_1,C_2 C_1,A_2 C_1,B_2)
\draw[thick,->](0,0)--node[below]{$\vec{v}$}(1,0);
\draw[thick,->](0,0)--node[above]{$\vec{u}$}(40:1);

\tkzInterLL(A_1,B_2)(A_2,B_1) \tkzGetPoint{P};
\tkzInterLL(A_1,C_2)(A_2,C_1) \tkzGetPoint{Q};
\tkzInterLL(C_1,B_2)(C_2,B_1) \tkzGetPoint{R};
\tkzDrawPoints(P,Q,R)
\tkzLabelPoints[right](R)
\tkzLabelPoints[left](P)
\tkzLabelPoints[above](Q)

\end{tikzpicture}
    \caption*{第4题}
\end{figure}


\item 在$\triangle ABC$的外面作正方形$ABEF$和$ACGH$, 又设$D$为
$\Vec{BC}$的中点,求证:
\begin{enumerate}
    \item $\Vec{AF}\cdot \Vec{AH}=\Vec{AB}\cdot \Vec{AC}$
    \item $BH\bot CF$且$\overline{BH}=\overline{CF}$
    \item $AD\bot FH$且$\overline{AD}=\frac{1}{2} \overline{FH}$
\end{enumerate}

\item 已知四边形$ABCD$内接于圆且$AC\bot BD$于$E$, $F$是边
$\Vec{BC}$的中点,求证:$EF\bot AD$
\item 已知$O$、$M$、$H$三点分别是$\triangle ABC$的外心,重心和
垂心,求证:$O$、$M$、$H$三点共线且$\overline{OM}=\frac{1}{2}\overline{MH}$.
\item 求证:连结四面体的一个顶点和这个顶点所对的面的重
心的四条线段交于同一点,且这交点分线段的比例都
是3:1.
\item 求证平行六面体的四条对角线相交于一点。
\item 在四面体$ABCD$中,如果$AB\bot DC$且$AD\bot BC$, 试
证明:
\[|\Vec{AB}|^2+|\Vec{DC}|^2=|\Vec{AD}|^2+|\Vec{BC}|^2=|\Vec{AC}|^2+|\Vec{BD}|^2\]
\item 已知四面体$ABCD$, $G_1,G_2$分别是$\triangle ABC$和$\triangle ABD$的
重心,$M$是棱$CD$的中点,试确定过$G_1$、$G_2$、$M$三点的
平面与棱$AB$的交点的位置。
\item 已知正方体$ABCD-A_1B_1C_1D_1$的棱长为1, $E$、$F$分别
是棱$\overline{BC}$, $\overline{CC_1}$的中点,求下列各异面直线的距离。
\begin{multicols}{3}
    \begin{enumerate}
        \item $AA_1$与$BD_1$
        \item $AC$与$BD_1$
        \item $AC$与$EF$
    \end{enumerate}
\end{multicols}
\end{enumerate}