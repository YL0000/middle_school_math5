
\chapter{空间解析几何初步}
\section{空间向量的坐标运算}
\subsection{空间直角坐标系与向量运算}
任取一点$O$(图8.1), 一个单位长,通过$O$点建立
三条互相垂直的数轴,$X$轴、$Y$轴、$Z$轴,并且使这三个数
轴的正方向构成右手系。这样我们
就说在空间建立了一个空间右手坐
标系,并用$OXYZ$来表示。$O$点
叫做坐标系的原点。$X$轴、$Y$轴、
$Z$轴总称为坐标轴。三个坐标轴每
两个决定一平面叫做坐标平面。坐标平面共有三个$OXY$、$OYZ$、
$OZX$,它们互相垂直并且把空间分为八个区域,每个区域叫做一个\textbf{卦限}。

\begin{figure}[htp]\centering
    \begin{minipage}[t]{0.48\textwidth}
    \centering
\begin{tikzpicture}[>=latex, scale=1]
\draw[<->](0,3.5)node[right]{$Z$}--(0,0)node [below right]{$O$}--(3,0)node[right]{$Y$};  
\draw[dashed](-2,0)--(0,0)--(1.5,1.5);
\draw[dashed](0,0)--(0,-1);
\draw[->](0,0)--(-1.5,-1.5)node[right]{$X$};
    \end{tikzpicture}
    \caption{}
    \end{minipage}
    \begin{minipage}[t]{0.48\textwidth}
    \centering
    \begin{tikzpicture}[>=latex, scale=1]
\draw[<->](0,3.5)node[right]{$Z$}--(0,0)--(3,0)node[right]{$Y$};  
\draw[->](0,0)--(-1.5,-1.5)node[left]{$X$};
\tkzDefPoints{0/0/O, 2/0/B, 2/2.5/P', 0/2.5/C, -1/-1/A}
\tkzDefPointsBy[translation= from O to A](B,P',C){B',P,C'}
\tkzDrawPolygon(B',P,C',A)
\tkzDrawPolygon(B,P',C,O)
\tkzDrawSegments(P,P' C,C' B,B')
\tkzLabelPoints[below](A,O,B)
\tkzLabelPoints[right](P)
\tkzLabelPoints[left](C)
    \end{tikzpicture}
    \caption{}
    \end{minipage}
    \end{figure}

设$P$是空间中任一点,通过$P$点作平面分别与坐标平面
$OYZ$、$OZX$、$OXY$平行(图8.2),并且分别与$X$
轴、$Y$轴、$Z$轴相交于$A$、$B$、$C$三点,如果$A$、$B$、$C$在
各坐标轴上的坐标分别为$x$、$y$、$z$, 则这三个有序实数组
$(x,y,z)$叫$P$点的\textbf{空间坐标}。简称坐标。$P$
点的坐标是$(x,y,z)$, 
记作$P(x,y,z)$. $x$、$y$、$z$分别叫做$P$点
的$X$坐标,$Y$坐标,$Z$坐标。

\begin{figure}[htp]
    \centering
\begin{tikzpicture}[>=latex]
    \draw[->](0,2.6)--(0,4)node[right]{$Z$};
    \draw[->](0,0)--(4.5,0)node[right]{$Y$};  
    \draw[->](0,0)--(-1.5,-1.5)node[left]{$X$};
\draw[->, very thick](0,0)--node[right]{$\eZ$}(0,1);
\draw[->, very thick](0,0)--(1,0)node[below]{$\eY$};
\draw[->, very thick](0,0)--(-.5,-.5)node[right]{$\eX$};
\node at (0,0)[below right]{$O$};
\tkzDefPoints{-1/1/A, 2/1/B, 3.3/1.75/C, .3/1.75/D, -1/2/A'}
\tkzDefPointsBy[translation = from A to A'](B,C,D){B',C',D'}
\tkzDrawPolygon[thick](A',B',C',D')
\tkzDrawPolygon[dashed](A,B,C,D)
\tkzDrawSegments[dashed](D,D')
\tkzDrawSegments[thick](A,A' B,B'  A,B)
\draw[dashed](0,1)--(0,2.6);
\draw[very thick, ->](B)--node[below]{$a_x\eX$}(C);
\draw[very thick, ->](C)--node[right]{$a_z\eZ$}(C');
\draw[very thick, ->](C')--node[above]{$a_y\eY$}(D');
\draw[thick, ->, dashed](B)--node[below]{$\vec{a}$}(D');
\end{tikzpicture}
    \caption{}
\end{figure}


如果沿$X$轴、$Y$轴、$Z$轴的正方向分别引单位向量$\eX$、$\eY$、$\eZ$(图8.3), 那么对空间任一向量$\vec{a}$, 存在唯一的有序数组
$(a_x,a_y,a_z)$使
\[\vec{a}=a_x\eX+a_y\eY+a_z\eZ\]
$(a_x,a_y,a_z)$就叫做$\vec{a}$在
$OXYZ$中的坐标。并简记作
\[\vec{a}=(a_x,a_y,a_z)\]
其中$a$叫做$\vec{a}$在$X$轴上的坐标分量。
$a_y$叫做$\vec{a}$在$Y$轴上的坐标分量。$a_z$叫做$\vec{a}$在$Z$轴上的坐标分量。

如果$\vec{a}=a_x\eX+a_y\eY+a_z\eZ$,那么分别对这个表示式两
边对$\eX,\eY,\eZ$取内积运算,就可得到
\[\begin{split}
    a_x&=\eX\cdot \vec{a}=|\vec{a}|\cos\langle \eX,\vec{a} \rangle \\
    a_y&=\eY\cdot \vec{a}=|\vec{a}|\cos\langle \eY,\vec{a} \rangle \\
    a_z&=\eZ\cdot \vec{a}=|\vec{a}|\cos\langle \eZ,\vec{a} \rangle \\
\end{split}\]

如果$\langle \eX,\vec{a} \rangle=\alpha$, $\langle \eY,\vec{a} \rangle=\beta$, $\langle \eZ,\vec{a} \rangle=\gamma$, 那
么$\alpha$、$\beta$、$\gamma$确定了$\vec{a}$在空间中的方向。$\alpha$、$\beta$、$\gamma$叫做
$\vec{a}$的方向角,$\cos\alpha$、$\cos\beta$、$\cos\gamma$叫做$\vec{a}$的方向余弦,于是
$\vec{a}$的单位向量
\[\vec{a}_0=(\cos\alpha, \cos\beta, \cos\gamma)\]

对空间任一点$P$, 它被相对于$O$点的位置向量所唯一确
定(图8.4)。设
\[\Vec{OP}=x\eX+y\eY+z\eZ\]
由上述点的坐标和向量坐标的定
义,$\Vec{OP}$的坐标$(x,y,z)$
也就是$P$点的坐标;反之$P$点的
坐标也是$\Vec{OP}$的坐标。由此可
见,给定了原点$O$和三个互相垂
直且构成右手系的单位向量$\eX,\eY,\eZ$,坐标系$OXYZ$也就完全确定了。因此,坐标系
$OXYZ$也可用$[O:\eX,\eY,\eZ]$来表示,$\eX,\eY,\eZ$叫
做$OXYZ$的基向量。

\begin{figure}[htp]
    \centering
\begin{tikzpicture}[>=latex]
\tkzDefPoints{0/0/A, 2/0/B, 2/2.5/C, 0/2.5/D, -.8/-.8/A'}
\tkzDefPointsBy[translation= from A to A'](B,C,D){B',C',D'}
\tkzDrawPolygon(A',B',C',D')
\tkzDrawPolygon[dashed](A,B,C,D)
\tkzDrawSegments(B,C C,D B,B' C,C' D,D')
\tkzDrawSegments[dashed](A,B A,D  A,A')
\draw[->](A')--(-1.5,-1.5)node[left]{$X$};
\draw[->](B)--(3,0)node[right]{$Y$};
\draw[->](D)--(0,3.5)node[right]{$Z$};
\draw[->, dashed](A)--(C');
\draw[->, very thick](0,0)--(0,1)node[left]{$\eZ$};
\draw[->, very thick](0,0)--node[above]{$\eY$}(1,0);
\draw[->, very thick](0,0)--(-.5,-.5)node[above]{$\eX$};
\node at (0,0)[below right]{$O$};
\node at (C')[right]{$P$};
\end{tikzpicture}
    \caption{}
\end{figure}

已知$A(x_1,y_1,z_1)$, $B(x_2,y_2,z_2)$, 则:
\[\begin{split}
   \Vec{AB}&=\Vec{OB}-\Vec{OA}\\
&=x_2\eX+y_2\eY+z_2\eZ-(x_1\eX+y_1\eY+z_1\eZ)\\
&=(x_2-x_1)\eX+(y_2-y_1)\eY+(z_2-z_1)\eZ\\
&=(x_2-x_1, y_2-y_1, z_2-z_1)
\end{split}\]
这就是说\textbf{一个向量的坐标,等于表示它的有向线段终点的坐
标减去起点的坐标}。例如,已知$A(2,-1,5)$、$B(3,
2,-7)$, 则
\[\Vec{AB}=[3-2,\; 2-(-1),\; -7-5]=(1,3,-12)\]

\begin{ex}
\begin{enumerate}
    \item 问在$OXYZ$中,哪个坐标平面与$X$轴垂直,哪个坐标
    平面与$Y$轴垂直,哪个坐标平面与$Z$轴垂直?
    \item 写出点$P(2,4,3)$在$OXYZ$的三个坐标平面上投
    影点的坐标。
    \item 求点$P(3,5,4)$关于坐标平面$OXY$的对称点的坐
    标。
    \item 点$P$在$OXYZ$中的坐标平面$OXY$上,若$P$点在平面
    直角坐标系$OXY$中的坐标是$(2,3)$, 求它在
    $OXYZ$中的坐标。
    \item 写出基向量$\eX,\eY,\eZ$的坐标。
    \item 已知$\vec{a}=12$, $\langle\eX ,\vec{a}\rangle=30^{\circ}$, $\langle\eY ,\vec{a}\rangle=45^{\circ}$, $\langle\eZ ,\vec{a}\rangle=60^{\circ}$,
求$\vec{a}$的坐标。
    \item 已知$P(-3,2,4)$, $Q(5,7,-2)$, 求$\Vec{PQ}$与$\Vec{QP}$的坐标。
    \item 已知$A(2,-1,5)$, $B(3,2,-1)$用基向量$\eX,\eY,\eZ$表示向量$\Vec{AB}$.
\end{enumerate}
\end{ex}

\subsection{向量的坐标运算}

\begin{blk}{定理}
     如果$\vec{a}=(a_x,a_y,a_z)$, $\vec{b}=(b_x, b_y,b_z)$, 
$\vec{c}=(c_x,c_y,c_z)$, 那么
\[\begin{split}
    \vec{a}\pm \vec{b}&=(a_x,a_y,a_z)\pm (b_x,b_y,b_z)
=(a_x\pm b_x,a_y\pm b_y,a_z\pm b_z)\\
\lambda\vec{a}&=\lambda(a_x, a_y,a_z)=(\lambda a_x,\lambda 
a_y,\lambda a_z)\\
\vec{a}\cdot \vec{b}&=(a_x, a_y, a_z)\cdot (b_x,b_y,b_z)
=a_xb_x+a_yb_y+a_zb_z\\
\end{split}\]
\[\vec{a}\x \vec{b}=\begin{vmatrix}
  \eX&\eY&\eZ\\
  a_x&a_y&a_z\\
  b_x&b_y&b_z  
\end{vmatrix},\qquad \left(\vec{a},\vec{b},\vec{c}\right)=\begin{vmatrix}
    a_x&a_y&a_z\\
    b_x&b_y&b_z\\  
    c_x&c_y&c_z
\end{vmatrix}\]
\end{blk}


证明留给同学作为练习。

下面我们研究如何用向量的坐标来表示向量垂直、平行
与共面的条件。

已知$\vec{a}\parallel \vec{b}\quad (\vec{b}\ne 0)$的充要条件是存在一实数$\lambda$,使
$$\vec{a}=\lambda\vec{b}$$
如果
$\vec{a}=(a_x,a_y,a_z)$, $\vec{b}=(b_x,b_y,b_z)$, 那么上面
条件用坐标表示,即为
\begin{equation}
    a_x=\lambda b_x,\qquad  a_y=\lambda b_y,\qquad  a_z=\lambda b_z
\end{equation}
或
\begin{equation}
    a_x:b_x=a_y:b_y=a_z:b_z
\end{equation}
这就是说\textbf{两个向量平行的充要条件是它们的坐标成比例}。

已知$\vec{a}\bot \vec{b}\quad \Longleftrightarrow \quad \vec{a}\cdot \vec{b}=0$
用坐标表示,即为
\begin{equation}
    \vec{a}\bot \vec{b}\quad \Longleftrightarrow \quad a_xb_x+a_yb_y+a_zb_z=0
\end{equation}

已知$\vec{a}=(a_x,a_y,a_z)$, $\vec{b}=(b_x,b_y,b_z)$, $\vec{c}=(c_x,c_y,c_z)$, 则
\[\vec{a}, \vec{b}, \vec{c}\text{ 共面} \quad \Longleftrightarrow \quad  (\vec{a}, \vec{b}, \vec{c})=0\]
即
\[\vec{a}, \vec{b}, \vec{c}\text{ 共面} \quad \Longleftrightarrow \quad \begin{vmatrix}
    a_x&a_y&a_z\\b_x&b_y&b_z\\c_x&c_y&c_z
\end{vmatrix}=0\]

\begin{example}
    已知 $\vec{a}=(1,1,1)$, $\vec{b}=(3,-1,2)$, 
$\vec{c}=(1,-3,0)$。
求证:$\vec{a},\vec{b},\vec{c}$共面。
\end{example}

\begin{solution}
\[\because\quad (\vec{a},\vec{b},\vec{c})=\begin{vmatrix}
    1&1&1\\3&-1&2\\1&-3&0
\end{vmatrix}=0\]

$\therefore\quad \vec{a},\vec{b},\vec{c}$共面。
\end{solution}


\begin{ex}
\begin{enumerate}
    \item 已知$\vec{a}=(-1,2,3)$, $\vec{b}=(2,-4,-6)$, 求证:
$\vec{a}\parallel \vec{b}$.
\item 试证下面各对向量线性相关。
\begin{enumerate}
    \item $\vec{a}=(2,-1,-2),\qquad \vec{b}=(6,-3,-6)$
    \item $\vec{a}=(-3,-5,4),\qquad   \vec{b}=(6,10,-8)$
\end{enumerate}

 \item 已知$\vec{a}=(2,3,4)$, $\vec{b}=(-3,-6,6)$, 求证$\vec{a}\bot \vec{b}$.

 \item 设$\vec{a}=(2,-1,4)$, $\vec{b}=(-4,-5.-1)$, 求使
 $\vec{a}-k\vec{b}$垂直于$\vec{b}$的实数$k$的值。

 \item 已知$\vec{a}=(-5,2,3)$, $\vec{b}=(0,-3,2)$, $\vec{c}=(5,
 -2,-3)$, 求证:$\vec{a},\vec{b},\vec{c}$三个向量共面。
\item 已知$P(x,y,z)$, $P_1(x_1,y_1,z_1)$, $P_2(x_2,y_2,
z_2)$, $P_3(x_3, У_3,z_3)$. 求证这四点共面的充要条件是
\[\begin{vmatrix}
    x-x_1 &y-y_1&z-z_1\\
    x_2-x_1 &y_2-y_1&z_2-z_1\\
    x_3-x_1 &y_3-y_1&z_3-z_1\\
\end{vmatrix}=0\quad \text{或}\quad \begin{vmatrix}
    x&y&z&1\\x_1 & y_1&z_1& 1\\
    x_2 & y_2&z_2& 1\\x_3 & y_3&z_3& 1
\end{vmatrix}=0\]
\end{enumerate}   
\end{ex}

\subsection{空间解析几何的基本问题}
\begin{blk}{问题1}
    求有向线段定比分点的坐标。
\end{blk}
 
已知$P_1(x_1,y_1,z_1)$、$P_2(x_2,y_2,z_2)$, 如果
$P(x,y,z)$, 按定比$\mu$分割$\Vec{P_1P_2}$, 那么
\[\Vec{OP}=\frac{\Vec{OP_1}+\mu \Vec{OP_2}}{1+\mu}\]
换用坐标表示,即为
\begin{equation}
    \begin{cases}
        x=\frac{x_1+\mu x_2}{1+\mu}\\
        y=\frac{y_1+\mu y_2}{1+\mu}\\
        z=\frac{z_1+\mu z_2}{1+\mu}\\
    \end{cases}
\end{equation}
(8.4)式就是求$\Vec{P_1P_2}$的\textbf{定比分点坐标的计算公式}。当
$\mu=1$时,$P$点是$\overline{P_1P_2}$的中点,$P$点的坐标是
\begin{equation}
    \begin{cases}
        x=\frac{x_1+ x_2}{2}\\
        y=\frac{y_1+ y_2}{2}\\
        z=\frac{z_1+ z_2}{2}\\
    \end{cases}
\end{equation}
公式(8.5)又叫做中点公式。


\begin{example}
已知$A(-1,2,2)$, $B(-4,2,5)$, 点$P$按定比
$\mu=2$分割$\Vec{AB}$, 求$P(x,y,z)$.
\end{example}

\begin{solution}
由于$\mu=2$,因此:
\[\begin{split}
    x&=\frac{-1+2\x (-4)}{1+2}=-3\\
    y&=\frac{2+2\x 2}{1+2}=2\\
    z&=\frac{2+2\x 5}{1+2}=4\\
\end{split}\]    
即:$P(-3,2,4)$
\end{solution}

\begin{blk}{问题2}
    求向量长度和两点间距离公式。
\end{blk}

若$\vec{a}=(a_x,a_y,a_z)$, 则
\begin{align}
    |\vec{a}|^2&=\vec{a}\cdot \vec{a} =a^2_x+a^2_y+a^2_z\nonumber\\
|\vec{a}|&=\sqrt{a^2_x+a^2_y+a^2_z}
\end{align}
(8.6)式就是求\textbf{向量$\vec{a}$的长度的计算公式}。

若$A(x_1,y_1,z_1)$、$B(x_2,y_2,z_2)$,则:
\begin{equation}
   |\Vec{AB}|=\sqrt{(x_2-x_1)^2+(y_2-y_1)^2+(z_2-z_1)^2} 
\end{equation}
(8.7)式就是求空间任意\textbf{两点间的距离公式}。


\begin{blk}
    {问题3} 求一向量的方向余弦。
\end{blk}

若$\vec{a}=(a_x,a_y,a_z)$, $\alpha,\beta,\gamma$为$\vec{a}$的方向角,则:
\[a_x=|\vec{a}|\cos\alpha,\qquad a_y=|\vec{a}|\cos\beta,\qquad a_z=|\vec{a}|\cos\gamma\]
于是得:
\begin{equation}
    \begin{cases}
    \cos\alpha=\frac{a_x}{|\vec{a}|}=\frac{a_x}{\sqrt{a^2_x+a^2_y+a^2_z}}\\
    \cos\beta=\frac{a_y}{|\vec{a}|}=\frac{a_y}{\sqrt{a^2_x+a^2_y+a^2_z}}\\
    \cos\gamma=\frac{a_z}{|\vec{a}|}=\frac{a_z}{\sqrt{a^2_x+a^2_y+a^2_z}}\\        
    \end{cases}
\end{equation}
(8.8)式就是\textbf{向量$\vec{a}$的方向余弦的计算公式}。

把(8.7)式两边平方加起来,得
\[\cos^2\alpha+\cos^2\beta+\cos^2\gamma=1\]
这就是说,\textbf{任何一个向量的方向余弦的平方和恒等于1}.

若$\vec{a}=(a_x,a_y,a_z)$,则$\vec{a}$的单位向量
\[\begin{split}
    \vec{a}_0=\frac{\vec{a}}{|\vec{a}|}&=\left(\frac{a_x}{|\vec{a}|},\frac{a_y}{|\vec{a}|},\frac{a_z}{|\vec{a}|}\right)\\
    &=(\cos\alpha,\cos\beta,\cos\gamma)
\end{split}\]
这就是说,\textbf{空间任一向量的单位向量的坐标分量正好等于它
的方向余弦}。

\begin{example}
    求$\vec{a}=(2,-3,1)$的方向余弦和它的单位向
量$\vec{a}_0$的坐标。
\end{example}

\begin{solution}
由于:$|\vec{a}|=\sqrt{2^2+(-3)^2+1^2}=\sqrt{14}$

$\therefore\quad \cos\alpha=\frac{2}{\sqrt{14}},\qquad \cos\beta=\frac{-3}{\sqrt{14}},\qquad \cos\gamma=\frac{1}{\sqrt{14}}$
\[\vec{a}_0=\left(\frac{2}{\sqrt{14}},\frac{-3}{\sqrt{14}},\frac{1}{\sqrt{14}}\right)\]
\end{solution}

\begin{blk}
    {问题4}
求两个向量的夹角。
\end{blk}

如果$\vec{a}=(a_x,a_y,a_z)$, $\vec{b}=(b_x,b_y,b_z)$,那么
\begin{equation}
    \begin{split}
\cos\langle \vec{a},\vec{b}\rangle &=\frac{\vec{a}\cdot \vec{b}}{|\vec{a}||\vec{b}|}\\
&=\frac{a_xb_x+a_yb_y+a_zb_z}{\sqrt{a^2_x+a^2_y+a^2_z}\cdot \sqrt{b^2_x+b^2_y+b^2_z}}
    \end{split}
\end{equation}
公式(8.9)就是求向量夹角的计算公式。



\begin{example}
    已知$\vec{a}=(1,1,0)$, $\vec{b}=(1,0,1)$, 
求$\langle \vec{a},\vec{b}\rangle$.
\end{example}

\begin{solution}
\[\cos\langle \vec{a}\cdot \vec{b} \rangle=\frac{1\x 1+1\x 0+0\x 1}{\sqrt{1^2+1^2+0^2}\cdot \sqrt{1^2+1^2+0^2}}=\frac{1}{2} \]
$\therefore\quad \langle \vec{a},\vec{b}\rangle=\frac{\pi}{3}$    
\end{solution}

\begin{ex}
\begin{enumerate}
    \item 已知$A(3,5,-7)$, $B(-2,4,3)$, 点$P$按定
    比$\mu=-2$分割$\Vec{AB}$, 求$P$点的坐标。
    \item 已知$P(3,-4,1)$, $Q(0,2,-3)$, 点$A$按定
    比$\mu=2$分割$\Vec{QP}$, 求$A$点的坐标。
    \item 已知$A(0,-1,1)$, $B(2,1,-3)$, 求$\overline{AB}$中点的坐
    标。
    \item 已知$\triangle ABC$的两个顶点$A(-4,-1,2)$, $B(3,
    5,-16)$, $\overline{AC}$边的中点在$Y$轴上,
    $\overline{BC}$边的中点在
    $OZX$平面上,求第三顶点$C$的坐标。
    \item 已知$A(3,-1,0)$、$B(2,1,-3)$, 求$A$、$B$
    两点间的距离。
    \item 已知$|\vec{a}|=10$, $\langle \eX,\vec{a} \rangle=60^{\circ}$, $\langle \eY,\vec{a} \rangle=60^{\circ}$

    求$\langle \eZ,\vec{a} \rangle=60^{\circ}$和$\vec{a}$的坐标。
\item 已知$\vec{a}\parallel \vec{b}$, $|\vec{a}|=10$, $\vec{b}=(3,-3,3)$, 求$\vec{a}$的坐标。

\item 已知$\vec{a}=(-1,2,3)$, $\vec{b}=(2,5,4)$, 求$\langle \vec{a},\vec{b} \rangle$

\item 已知$\vec{a}=(2,-3,5)$, $\vec{b}=(-4,2,6)$, 求证
$\vec{a}\nparallel \vec{b}$.
\end{enumerate}
\end{ex}

\subsection{习题8.1}

\begin{enumerate}
    \item 如果向量$\vec{a}$、$\vec{b}$、$\vec{c}$分别平行于$X$轴、$Y$轴、$Z$轴,问
    它们的坐标各有什么特点?
    \item 
    如果$\vec{a}$的$x$坐标是0, 那么$\vec{a}$与哪个平面平行。
    \item 已知$\vec{a}=(2,-1,3)$、$\vec{b}=(-3,0,4)$, 求满足下
    列关系的向量$\vec{c}$的坐标。
\begin{multicols}{2}
\begin{enumerate}
    \item $3\vec{a}+2\vec{c}=\vec{b}$
    \item $\vec{a}-3\vec{c}=2\vec{b}$
    \item $\vec{a}-2\vec{c}=3\vec{b}-\vec{c}$
    \item $2(3\vec{a}-\vec{c})+\vec{b}=0$
\end{enumerate}
\end{multicols}
    \item 已知$A(2,-1,7)$, $B(4,5,-2)$, 求每个坐标平
    面分割$\Vec{AB}$的比值。
    \item 已知$A(2,3,6)$, $B(5,2,8)$, 直线$AB$上有$C$点
    使$B$点为$\overline{AC}$的中点,求$C(x,y,z)$.
    \item 已知$A=(x_1,y_1,z_1)$、$B=(x_2,y_2,z_2)$、
    $C=(x_3,y_3,z_3)$, 求$\triangle ABC$的重心。

    \item 已知$\vec{a}=(1,2,-2)$, $\vec{b}=(3,4,2)$,
    $\vec{c}=(-2,-4,4)$, 求证:$\vec{a}$、$\vec{b}$、$\vec{c}$线性相关。

    \item 已知$A(4,1,3)$, $B(2,-5,1)$, $C(3,7,-5)$. 求向量$\Vec{AB},\Vec{BA},\Vec{AC},\Vec{BC}$的坐标和长度
    (精确到0.01)。
    \item 已知$A(1,1,\sqrt{2})$, 求$\Vec{OA}$与三个坐标轴的夹角。
    \item 已知$\vec{a}=(-1,1,0)$, $\vec{b}=(1,-2,2)$,
    求$\langle \vec{a},\vec{b}\rangle$.
\end{enumerate}

\section{空间的平面~~直线与球面方程}

\subsection{空间的平面方程}

已知非零向量$\vec{n}=(a,b,c)$和定点$P_0(x_0,y_0,z_0)$,
过$P_0$点作平面$\pi$与$\vec{n}$垂直,求平面
$\pi$的方程。

\begin{figure}[htp]
    \centering
\begin{tikzpicture}[>=latex]
\draw[->](0,0)node[below]{$O$}--(4,0)node[right]{$Y$};
\draw[->](0,0)--(0,4)node[right]{$Z$};
\draw[->](0,0)--(-1,-1)node[right]{$X$};
\tkzDefPoints{.5/2.5/A, 3/2/B, -.5/1/A', 0/0/O}
\tkzDefPointsBy[translation = from A to A'](B){B'}
\tkzDrawPolygon[fill=white](A,B,B',A')
\tkzDefPoint(30:2.6){P}
\tkzDefPoint(60:2){P_0}
\tkzDrawSegments[dashed, ->](O,P O,P_0)
\tkzDrawSegments[->](P_0,P)
\tkzLabelPoints[above](P_0,P)
\tkzDefPointWith[linear, K=.5](O,P) \tkzGetPoint{P1}
\tkzDefPointWith[linear, K=.43](O,P_0) \tkzGetPoint{P2}


\draw[->, thick](.65,2.2)--node[left]{$\vec{n}$}+(72:1);

\tkzDrawSegments (O,P1 O,P2)
\end{tikzpicture}
    \caption{}
\end{figure}

设$P(x,y,z)$为平面$\pi$上一动点,因为
$\Vec{P_0P}\bot\vec{n}$, 所以
$\Vec{P_0P}\cdot \vec{n}=0$,
即:
\begin{equation}
    \left(\Vec{OP}-\Vec{OP_0}\right)\cdot\vec{n}=0
\end{equation}
反之,如果$P(x,y,z)$满足(8.10)式,那么$P$点一定在
平面$\pi$上,所以(8.10)式就是\textbf{平面$\pi$的向量方程}。

(8.10)式用坐标表示即可写为
\begin{equation}
    a(x-x_0)+b(y-y_0)+c(z-z_0)=0
\end{equation}
(8.11)式就叫做\textbf{平面的点法向式方程}。其中$\vec{n}=(a,b,c)$, 
叫做平面$\pi$的一条法线向量。

如果令$d=-(ax_0+by_0+cz_0)$, 那么(8.11)式又可
写为
\begin{equation}
    ax+by+cz+d=0
\end{equation}
方程(8.12)又叫做\textbf{平面的普通方程},其中$a,b,c$至少有
一个不为零。

显然,如果$\vec{n}=(a,b,c)$是平面$\pi$的一个法线向
量,那么对任何非零常数$k$, $k\vec{n}$也是$\pi$的法线向量。这
样,若取$k\vec{n}$作为平面的法线向量,则$\pi$的方程还可写为
\[k(ax+by+cz+d)=0\]
因此,同一个平面方程,仅仅相差一个常数因子。

由方程(8.12)可以看出,平面的方程是$x,y,z$的一
次方程;反之,如果设$(x_0,y_0,z_0)$是三元一次方程
$ax+by+cz+d=0$
的一个解,则
\[ax_0+by_0+cz_0+d=0\]
两式相减,得
\begin{equation}
    a(x-x_0)+b(y-y_0)+c(z-z_0)=0
\end{equation}
如果建立空间直角坐标系,作$\Vec{OP_0}=(x_0,y_0,z_0)$,
 $\vec{n}=(a,b,c)$, 那么(8.13)式就是通过$P_0$且垂直于$\vec{n}$
的一个平面方程,这就是说,\textbf{任何一个三元一次方程都表示
一个平面}。这样,在空间解析几何中,一个平面和一个三元
一次方程是同一码事。

由以上分析,我们还可得到一个结论,即,\textbf{任给一个平
面$\pi:\; ax+by+cz+d=0$, 其中$x,y,z$的系数向量  
$\vec{n}=(a,b,c)$是平面$\pi$的一个法线向量。
}

\begin{example}
    求通过点$P(2,-1,3)$且垂直于$\vec{n}=(2,-1,5)$
的平面方程。
\end{example}

\begin{solution}
    由平面的点法式方程,得所求平面方程为
\[2(x-2)+(-1)[y-(-1)]+5(z-3)=0\]
整理得
\[2x-y+5z-20=0\]
\end{solution}




\begin{example}
    已知$A(x_1,y_1,z_1)$, $B(x_2,y_2,z_2)$, 
$C(x_3,y_3,z_3)$三点不共线。求通过$A$、$B$、$C$的平面方
程。
\end{example}

\begin{solution}
    设$P(x,y,z)$为所求平面的一个动点,则$P$点
与$A$、$B$、$C$三点共面的充要条件是
\[\begin{vmatrix}
    x-x_1&y-y_1&z-z_1\\
    x_2-x_1&y_2-y_1&z_2-z_1\\
    x_3-x_1&y_3-y_1&z_3-z_1\\
\end{vmatrix}=0\]
这就是\textbf{通过$A$、$B$、$C$三点的平面方程},叫做平面方程的三
点式。
\end{solution}




\begin{example}
    求通过原点和两点$(2,0,1)$, $(0,1,3)$
的平面方程。
\end{example}

\begin{solution}
\textbf{方法1:} 由平面方程的三点式,得
\[\begin{vmatrix}
x-0&y-0&z-0\\    
2-0&0-0&1-0\\ 
0-0&1-0&3-0\\ 
\end{vmatrix}=0\]
展开化简,得
\[x+6y-2z=0\]
\textbf{方法2:}
设所求的平面方程为$ax+by+cz+d=0$, 
把已知三点的坐标,代入上面方程,得
\[\begin{cases}
    d=0\\
    2a+c=0\\
    b+3c=0
\end{cases}\]
解此方程组,得
\[a=-\frac{1}{2}c,\qquad b=-3c,\qquad d=0\]
所以,所求的平面方程为
\[\frac{1}{2}cx-3cy+cz=0\]
即:$x+6y-2z=0$.
\end{solution}

\begin{example}
    求通过点$(1,2,3)$且平行于平面
$2x+y-z+3=0$的平面方程。
\end{example}

\begin{solution}
    已知平面的一个法线向量是$\vec{n}=(2,1,-1)$, 
它与所求平面垂直,由平面的点法向式方程,得所求方程为
\[2(x-1)+1(y-2)+(-1)(z-3)=0\]
整理,得
\[2x+y-z-1=0\]
\end{solution}

\begin{example}
    求点$P_1(x_1,y_1,z_1)$到平面$\pi:\; ax+by+cz+d=0$的距离$d$(图8.6).
\end{example}

\begin{figure}[htp]
    \centering
\begin{tikzpicture}[>=latex, scale=1.2]
\draw(0,0)--(3,0)--(3.5,1)--(.5,1)--(0,0);
\draw[->, thick](3,.5)--(3,1.5)node[right]{$\vec{n}$};
\tkzDefPoints{1/0/A, 1.5/1/B, 1.9/.5/P_0}
\tkzDefPointWith[linear, K=1.6](A,B)  \tkzGetPoint{P_1}
\tkzDefPointWith[linear, K=1.6](B,A)  \tkzGetPoint{S}
\tkzDefPointWith[linear, K=0.6](S,P_0)  \tkzGetPoint{S1}
\tkzDrawSegments[dashed](A,B P_0,S1)
\tkzDrawSegments(P_1,B S,A P_1,P_0 S,S1)
\tkzLabelPoints[right](P_1,P_0)
\end{tikzpicture}
    \caption{}
\end{figure}

\begin{solution}
    过$P_1$作$P_1P_0$垂直平面$\pi$
于$P_0$点,则
\[d=|\Vec{P_0P_1}|\]
设$P_0$的坐标为$(x_0,y_0,z_0)$. 则
\[|\Vec{P_0P_2}|=|\Vec{P_0P_1} \cdot \vec{n}_0|\]
其中$\vec{n}_0$是$\pi$的单位法向量。换用坐标表示,即为
\[|\Vec{P_0P_1}|=\frac{a(x_1-x_0)+b(y_1-y_0)+c(z_1-z_0)}{\sqrt{a^2+b^2+c^2}}\]
因为$P_0\in \pi$, 所以
\[ax_0+by_0+cz_0+d=0\]
其中: $d=-(ax_0+by_0+cz_0)$, 代入上式,得
\[|\Vec{P_0P_1}|=\frac{ax_1+by_1+cz_1+d}{\sqrt{a^2+b^2+c^2}}\]
\[d=\frac{|ax_1+by_1+cz_1+d|}{\sqrt{a^2+b^2+c^2}}\]
\end{solution}

例8.9说明,如果要求一点到一平面的距离,只要把这
一点的坐标代入平面方程。取绝对值,再除以系数向量的长
度就可求出。

\begin{ex}
\begin{enumerate}
    \item  求三个坐标平面的方程。
    \item  求过点$A(1,2,-3)$, 以$\vec{n}=(1,-3,2)$为法
    线向量的方程。
    \item  求过点$P_0(x_0,y_0,z_0)$且垂直于$X$轴的平面方程。
    \item  已知两点$A(2,3,4)$, $B(-2,4,3)$, 求$\overline{AB}$
    的垂直平分面的方程。
    \item  求通过点$P_0(x_0,y_0,z_0)$且平行于$OXY$平面的方程。
    \item  证明方程$ax+by+cz=0$, 是通过原点的平面,其中
    $a$、$b$、$c$至少有一个不为零
    \item  求过原点和两点$(1,0,-1)$, $(0,2,3)$的平
    面方程。
    \item  求过点$(3,5,-2)$且平行于平面$2x-y+3z=0$
    的平面方程。
    \item  求点$(3,-2,5)$到平面$3x-4y-z+3=0$的距
    离。
\end{enumerate}
\end{ex}


\subsection{空间的直线方程}
已知,一定点$P_0(x_0,y_0,z_0)$
和一向量$\vec{a}=(a_1,a_2,a_3)$, 求过
$P_0$且平行于向量$\vec{a}$的直线方程。

\begin{figure}[htp]
    \centering
\begin{tikzpicture}[>=latex]
\draw[->](0,0)node[below]{$O$}--(3.5,0)node[right]{$Y$};
\draw[->](0,0)--(0,3)node[right]{$Z$};
\draw[->](0,0)--(-1.25,-1.25)node[right]{$X$};
\draw[domain=-1:2.5, samples=10, thick]plot(\x, {1.5-\x})node[right]{$\ell$};

\tkzDefPoints{0/0/O, 0.6/0.9/P_0, -.5/2/P}
\tkzDrawSegments[->, thick](O,P O,P_0)
\draw[thick,->](0,0)--node[below]{$\vec{a}$}(135:1);
\tkzLabelPoints[above](P,P_0)
\end{tikzpicture}
    \caption{}
\end{figure}

设$P(x,y,z)$是所求直线
$\ell$上一动点,则存在一实数$t$使
\[\Vec{P_0P}=t\vec{a},\qquad \Vec{OP}=\Vec{OP_0}+t\vec{a}\]
换用坐标表示,即为
\begin{equation}
    \begin{cases}
        x=x_0+a_1t\\
y = y_0+a_2t\\
z=z_0+a_3t
    \end{cases}
\end{equation}
(8.14)式叫做直线$\ell$的\textbf{参数方程}。$t$叫做\textbf{参数}。

如果$a_1,a_2,a_3$都不为零,从(8.14)式消去参数$t$, 得
\[\frac{x-x_0}{a_1}=\frac{y-y_0}{a_2}=\frac{z-z_0}{a_3}\]
(8.15)式叫做$\ell$的\textbf{点、方向式方程}又叫\textbf{对称式方程}。其中
$\vec{a}=(a_1,a_2,a_3)$叫做$\ell$的方向向量。如果取$\vec{a}$的单位向量
\[\vec{a}_0=(\cos\alpha, \cos\beta,\cos\gamma)\]
作为方向向量,则$\ell$的方程为
\[\frac{x-x_0}{\cos\alpha}=\frac{y-y_0}{\cos\beta}=\frac{z-z_0}{\cos\gamma}\]
$\cos\alpha, \cos\beta,\cos\gamma$又叫做有向\textbf{直线$\ell$的方向余弦}。

如果直线$\ell$通过两点$P_1(x_1,y_1,z_1)$, $P_2(x_2,y_2,z_2)$, 
则直线$\ell$的方向向量可取
\[\Vec{P_1P_2}=(x_2-x_1,y_2-y_1,z_2-z_1)\]
这时直线$\ell$的方程可写为
\begin{equation}
    \frac{x-x_1}{x_2-x_1}=\frac{y-y_1}{y_2-y_1}=\frac{z-z_1}{z_2-z_1}
\end{equation}
方程(8.16)一般叫做直线的\textbf{两点式方程}。



\begin{example}
    求通过$P_0(1,-1,2)$, 且和向量$\vec{a}=(2,3,1)$平
行的直线$\ell$的方程。
\end{example}


\begin{solution}
    由直线的对称式方程可得直线$\ell$的方程为
\[\frac{x-1}{2}=\frac{y+1}{3}=z-2\]
\end{solution}

\begin{example}
    求通过两个不同点$P_1(x_1,y_1,z_1)$, $P_2(x_2,
y_2,z_2)$的直线的参数方程。
\end{example}


\begin{solution}
    取直线$P_1P_2$的方向向量
\[    \Vec{P_1P_2}=(x_2-x_1, y_2-У_1,z_2-z_1)\]
    由直线的参数方程。可得直线$P_1P_2$的参数方程为
\[\begin{cases}
    x=x_1+(x_2-x_1)t\\
    y=y_1+(y_2-y_1)t\\
    z=z_1+(z_2-z_1)t
\end{cases}\]
\end{solution}    

\begin{example}
    已知$P_1(5,0,1)$, $P_2(5,6,4)$, 求直线
$P_1P_2$的参数方程。
\end{example}


\begin{solution}
    由例8.11, 可知直线$P_1P_2$的参数方程为
    \[\begin{cases}
     x=5\\
y=6t\\
z=1+3t\\
    \end{cases}\]
\end{solution}

在例8.12中,由于$\Vec{P_1P_2}=(0,6,3)$, 其中$x$坐标为
零,因此直线$P_1P_2$不能写为对称式方程,但确能用参数方
程来表达。由此可看到,直线的参数方程比较优越。


\begin{ex}
\begin{enumerate}
    \item 求通过点$P_0(-1,2,-3)$且平行于向量$\vec{s}=(2,3,-5)$
    的直线方程。
    \item 求通过$P_0(2,3,1)$, $P_1(-1,-2,3)$的直线方程。
    \item 求通过点$(2,3,1)$且和$X$轴平行的直线方程。
    \item 求过点$(2,-3,7)$, 其方向向量为$(2,0,3)$的直
    线方程。
    \item 求直线$2x-6=4-y=2-5$的方向向量。
    \item 求平行于两平面$x-2y+5z+2=0$和$3x+y-z
    +5=0$的交线,且通过原点的直线方程。
\end{enumerate}
\end{ex}


\subsection{球面方程}
空间一动点$P(x,y,z)$在以$A(a,b,c)$为球
心,$R$为半径的球面上的充要条件是
\[|\Vec{OP}-\Vec{OA}|=R\]
或
\[(\Vec{OP}-\Vec{OA})\cdot (\Vec{OP}-\Vec{OA})=R^2\]
换用坐标表示,条件可写为
\begin{equation}
    (x-a)^2+(y-b)^2+(z-c)^2=R^2
\end{equation}

\begin{figure}[htp]
    \centering
\begin{tikzpicture}[>=latex]
\draw[<->](0,4)node[right]{$Z$}--(0,0)node[below right]{$O$}--(3.5,0)node[right]{$Y$};
\draw[->](0,0)--(-1,-1)node[right]{$X$};
\tkzDefPoints{1.5/1.75/A, 0/0/O, 1/2.62/P}

\draw[thick](A) circle (1);
\draw[dashed](A) ellipse[x radius=1, y radius=.4];
\draw[thick](2.5,1.75) arc [x radius=1, y radius=.4,start angle =0, end angle =-180];
\tkzDrawSegments[dashed, ->](O,P A,P O,A)
\tkzLabelPoints[right](A)
\tkzLabelPoints[above](P)
\end{tikzpicture}
    \caption{}
\end{figure}


方程(8.16)就是以$A(a,b,c)$为球心,以$R$为半径的\textbf{球面
方程}。当$A$点在原点,球面方程变为
\[    x^2+y^2+z^2=R^2\]



\begin{ex}
\begin{enumerate}
    \item 求以$A(1,2,-2)$为球心,3为半径的球面方程。
    \item 求球心在原点,半径等于5的球面方程。
    \item 设一动点$Q$在以$A(0,4,0)$为球心,2为半径的球面
    上变动,求$\overline{OQ}$中点的轨迹。
\end{enumerate}
\end{ex}

\subsection*{习题8.2}
\begin{enumerate}
    \item 求满足以下条件的平面方程。
\begin{enumerate}
\item 通过点$P_0(5,3,4)$且垂直于向量$\vec{n}=(1,1,1)$;
\item 通过坐标原点且垂直于$\vec{n}=\left(-\frac{1}{3},\frac{2}{3},-\frac{2}{3}\right)$;
\item 垂直于$\vec{n}=\left(\frac{1}{2},\frac{\sqrt{3}}{2},0\right)$且与原点的距离等于5.
\end{enumerate}

    \item 说出如下方程表示的平面的几何特征。
\begin{multicols}{3}
\begin{enumerate}
    \item $x=2$ \item $x=y$ \item $x+y+z=1$
\end{enumerate}
\end{multicols}

    \item 求证通过三点$A(a,0,0)$, $B(0,b,0)$, $C(0,
    0,c)$的平面方程为
\[\frac{x}{a}+\frac{y}{b}+\frac{z}{c}=1\] 

\item 如图,试写出长方体$ABC
D-A'B'C'D'$的各侧面,底面的平面方程以及各
棱所在的直线方程。已知
$\overline{AB}=a$, $\overline{AD}=b$, 
$\overline{AA'}=c$.

\begin{figure}[htp]
    \centering
\begin{tikzpicture}[>=latex, scale=1.3]
    \tkzDefPoints{0/0/A, -.5/-.5/B, 1/-.5/C, 1.5/0/D, 0/2/A'}
    \tkzDefPointsBy[translation= from A to A'](B,C,D){B',C',D'}
    \tkzDrawPolygon[thick](A',B',C',D')
    \tkzDrawPolygon[dashed](A,B,C,D)
    \tkzDrawSegments[thick](B,C C,D B,B' C,C' D,D')
    \tkzDrawSegments[dashed](A,B A,D  A,A')
    \draw[->](B)--(-1,-1)node[left]{$X$};
    \draw[->](D)--(2.5,0)node[right]{$Y$};
    \draw[->](A')--(0,3)node[right]{$Z$};
    \node at (0,0)[left]{$O$};
\tkzLabelPoints[below](B,C,A,D)
\tkzLabelPoints[left](B',A')
\tkzLabelPoints[right](C',D')
\end{tikzpicture}
    \caption*{第4题}
\end{figure}


\item 分别求两点$P_1(3,9,1)$, $P_2(4,1,5)$到平面$S:\;
x-2y+2z-3=0$的距离。
\item 求两条直线
\[\begin{split}
    \ell_1:&\quad \frac{x-1}{3}=\frac{y+2}{6}=\frac{z-5}{2}\\
    \ell_2:&\quad \frac{x}{2}=\frac{y-3}{9}=\frac{z+1}{6}\\
\end{split}\]
的夹角。
\item 求通过点$(1,-1,2)$并与已知平面:$x+y+z=1$
垂直的直线方程。并求这条直线与平面交点的坐标。
\item 在直线$\ell:\; x=1+2t,\; y=8+t,\; z=8+3t$上求
一点使它和原点的距离等于35.
\item 求满足下列条件的球面方程。
\begin{multicols}{2}
    \begin{enumerate}
    \item 球心在$(-2,3,-6)$, 半径是7
    \item 球心在$(4,0,0)$, 半径是2
    \item 球心在$(0,-4,3)$, 半径是5
    \item 球心在$(0,-5,0)$, 半径是2
    \item 球心在$\left(\frac{2}{3},-\frac{1}{3},0\right)$,
半径是1
\end{enumerate}
\end{multicols}

\item 求球面:$x^2+y^2+z^2+4x-6y-2z+5=0$的球心和
半径。
\end{enumerate}

\section*{复习题八}
\begin{enumerate}
\item 已知点$P(3,-1,2)$和$M(a,b,c)$, 求$P$、$M$两点
分别关于坐标平面、坐标轴以及原点的对称点的坐标。
\item 求点$P(2,5,6)$到坐标原点以及三条坐标轴的距离。
\item 已知$\Vec{OA}=(6,2,9)$,求$\Vec{OA}$与三个坐标平面的夹角.
\item 已知$A(-2,1,3)$, $B(0,-1,2)$,求与$A$、$B$两点
距离相等点的轨迹方程。
\item 已知$A(a,0,0)$, $B(0,b,0)$, $C(0,0,c)$, 原
点到平面$(A,B,C)$的距离为$d$, 求证
\[\frac{1}{a^2}+\frac{1}{b^2}+\frac{1}{c^2}=\frac{1}{d^2}\]
\item 在$Z$轴上求一点,使得到$A(-4,1,7)$, $B(3,5,-2)$两点的距离相等。
\item 已知四面体$ABCD$,且$A(x_1,y_1,z_1)$, $B(x_2,
y_2,z_2)$, $C(x_3,y_3, z_3)$, $D(x_4, y_4,z_4)$, 
求它的重心的坐标。
\item 已知$\vec{a}=(a_1,a_2,a_3)$, $\vec{b}=(b_1,b_2,b_3)$, 求证:以$\vec{a}$, $\vec{b}$为邻边的平行四边形面积
\[S=\sqrt{\begin{vmatrix}
    a_2&a_3\\b_2&b_3
\end{vmatrix}^2+\begin{vmatrix}
    a_3&a_1\\b_3&b_1
\end{vmatrix}^2+\begin{vmatrix}
    a_1&a_2\\b_1&b_2
\end{vmatrix}^2}\]

\item 已知点$P(1,3,5)$和$\vec{a}(-2,1,1)$. 求通过$P$点
具有方向$\vec{a}$的直线与平面$2x+3y-z=1$的交点。

\item 求通过如下三点的平面方程。
\begin{enumerate}
    \item $(2,1,1),\qquad (3,-1,1),\qquad (4,1,-1)$
    \item $(-2,3,-1),\qquad (2,2,3),\qquad (-4,-1,1)$
    \item $(-5,-1,-2),\qquad (1,2,-1),\qquad (3,-1,2)$
\end{enumerate}

\item 求下列通过已知点$P$且以$\vec{a}$为方向向量的直线方程。
\begin{multicols}{2}
    \begin{enumerate}
    \item $P(2,1,3),\quad \vec{a}=(1,1,-2)$
    \item $P(-5,3,4),\quad \vec{a}=(-2,2,1)$
    \item $P(4,-3,2),\quad \vec{a}=(5,0,3)$
    \item $P(0,0,0),\quad \vec{a}=(2,-3,5)$
    \item $P(a,b,c),\quad \vec{a}=(\ell,m,n)$
\end{enumerate}
\end{multicols}

\end{enumerate}

