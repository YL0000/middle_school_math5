
\chapter{空间解析几何初步}
\section{空间向量的坐标运算}
\subsection{空间直角坐标系与向量运算}
任取一点$O$(图8.1), 一个单位长,通过$O$点建立
三条互相垂直的数轴,$X$轴、$Y$轴、$Z$轴,并且使这三个数
轴的正方向构成右手系。这样我们
就说在空间建立了一个空间右手坐
标系,并用$OXYZ$来表示。$O$点
叫做坐标系的原点。$X$轴、$Y$轴、
$Z$轴总称为坐标轴。三个坐标轴每
两个决定一平面叫做坐标平面。坐标平面共有三个$OXY$、$OYZ$、
$OZX$,它们互相垂直并且把空间分为八个区域,每个区域叫做一个\textbf{卦限}。

\begin{figure}[htp]\centering
    \begin{minipage}[t]{0.48\textwidth}
    \centering
\begin{tikzpicture}[>=latex, scale=1]
\draw[<->](0,3.5)node[right]{$Z$}--(0,0)node [below right]{$O$}--(3,0)node[right]{$Y$};  
\draw[dashed](-2,0)--(0,0)--(1.5,1.5);
\draw[dashed](0,0)--(0,-1);
\draw[->](0,0)--(-1.5,-1.5)node[right]{$X$};
    \end{tikzpicture}
    \caption{}
    \end{minipage}
    \begin{minipage}[t]{0.48\textwidth}
    \centering
    \begin{tikzpicture}[>=latex, scale=1]
\draw[<->](0,3.5)node[right]{$Z$}--(0,0)--(3,0)node[right]{$Y$};  
\draw[->](0,0)--(-1.5,-1.5)node[left]{$X$};
\tkzDefPoints{0/0/O, 2/0/B, 2/2.5/P', 0/2.5/C, -1/-1/A}
\tkzDefPointsBy[translation= from O to A](B,P',C){B',P,C'}
\tkzDrawPolygon(B',P,C',A)
\tkzDrawPolygon(B,P',C,O)
\tkzDrawSegments(P,P' C,C' B,B')
\tkzLabelPoints[below](A,O,B)
\tkzLabelPoints[right](P)
\tkzLabelPoints[left](C)
    \end{tikzpicture}
    \caption{}
    \end{minipage}
    \end{figure}

设$P$是空间中任一点,通过$P$点作平面分别与坐标平面
$OYZ$、$OZX$、$OXY$平行(图8.2),并且分别与$X$
轴、$Y$轴、$Z$轴相交于$A$、$B$、$C$三点,如果$A$、$B$、$C$在
各坐标轴上的坐标分别为$x$、$y$、$z$, 则这三个有序实数组
$(x,y,z)$叫$P$点的\textbf{空间坐标}。简称坐标。$P$
点的坐标是$(x,y,z)$, 
记作$P(x,y,z)$. $x$、$y$、$z$分别叫做$P$点
的$X$坐标,$Y$坐标,$Z$坐标。

\begin{figure}[htp]
    \centering
\begin{tikzpicture}[>=latex]
    \draw[->](0,2.6)--(0,4)node[right]{$Z$};
    \draw[->](0,0)--(4.5,0)node[right]{$Y$};  
    \draw[->](0,0)--(-1.5,-1.5)node[left]{$X$};
\draw[->, very thick](0,0)--node[right]{$\eZ$}(0,1);
\draw[->, very thick](0,0)--(1,0)node[below]{$\eY$};
\draw[->, very thick](0,0)--(-.5,-.5)node[right]{$\eX$};
\node at (0,0)[below right]{$O$};
\tkzDefPoints{-1/1/A, 2/1/B, 3.3/1.75/C, .3/1.75/D, -1/2/A'}
\tkzDefPointsBy[translation = from A to A'](B,C,D){B',C',D'}
\tkzDrawPolygon[thick](A',B',C',D')
\tkzDrawPolygon[dashed](A,B,C,D)
\tkzDrawSegments[dashed](D,D')
\tkzDrawSegments[thick](A,A' B,B'  A,B)
\draw[dashed](0,1)--(0,2.6);
\draw[very thick, ->](B)--node[below]{$a_x\eX$}(C);
\draw[very thick, ->](C)--node[right]{$a_z\eZ$}(C');
\draw[very thick, ->](C')--node[above]{$a_y\eY$}(D');
\draw[thick, ->, dashed](B)--node[below]{$\vec{a}$}(D');
\end{tikzpicture}
    \caption{}
\end{figure}


如果沿$X$轴、$Y$轴、$Z$轴的正方向分别引单位向量$\eX$、$\eY$、$\eZ$(图8.3), 那么对空间任一向量$\vec{a}$, 存在唯一的有序数组
$(a_x,a_y,a_z)$使
\[\vec{a}=a_x\eX+a_y\eY+a_z\eZ\]
$(a_x,a_y,a_z)$就叫做$\vec{a}$在
$OXYZ$中的坐标。并简记作
\[\vec{a}=(a_x,a_y,a_z)\]
其中$a$叫做$\vec{a}$在$X$轴上的坐标分量。
$a_y$叫做$\vec{a}$在$Y$轴上的坐标分量。$a_z$叫做$\vec{a}$在$Z$轴上的坐标分量。

如果$\vec{a}=a_x\eX+a_y\eY+a_z\eZ$,那么分别对这个表示式两
边对$\eX,\eY,\eZ$取内积运算,就可得到
\[\begin{split}
    a_x&=\eX\cdot \vec{a}=|\vec{a}|\cos\langle \eX,\vec{a} \rangle \\
    a_y&=\eY\cdot \vec{a}=|\vec{a}|\cos\langle \eY,\vec{a} \rangle \\
    a_z&=\eZ\cdot \vec{a}=|\vec{a}|\cos\langle \eZ,\vec{a} \rangle \\
\end{split}\]

如果$\langle \eX,\vec{a} \rangle=\alpha$, $\langle \eY,\vec{a} \rangle=\beta$, $\langle \eZ,\vec{a} \rangle=\gamma$, 那
么$\alpha$、$\beta$、$\gamma$确定了$\vec{a}$在空间中的方向。$\alpha$、$\beta$、$\gamma$叫做
$\vec{a}$的方向角,$\cos\alpha$、$\cos\beta$、$\cos\gamma$叫做$\vec{a}$的方向余弦,于是
$\vec{a}$的单位向量
\[\vec{a}_0=(\cos\alpha, \cos\beta, \cos\gamma)\]

对空间任一点$P$, 它被相对于$O$点的位置向量所唯一确
定(图8.4)。设
\[\Vec{OP}=x\eX+y\eY+z\eZ\]
由上述点的坐标和向量坐标的定
义,$\Vec{OP}$的坐标$(x,y,z)$
也就是$P$点的坐标;反之$P$点的
坐标也是$\Vec{OP}$的坐标。由此可
见,给定了原点$O$和三个互相垂
直且构成右手系的单位向量$\eX,\eY,\eZ$,坐标系$OXYZ$也就完全确定了。因此,坐标系
$OXYZ$也可用$[O:\eX,\eY,\eZ]$来表示,$\eX,\eY,\eZ$叫
做$OXYZ$的基向量。

\begin{figure}[htp]
    \centering
\begin{tikzpicture}[>=latex]
\tkzDefPoints{0/0/A, 2/0/B, 2/2.5/C, 0/2.5/D, -.8/-.8/A'}
\tkzDefPointsBy[translation= from A to A'](B,C,D){B',C',D'}
\tkzDrawPolygon(A',B',C',D')
\tkzDrawPolygon[dashed](A,B,C,D)
\tkzDrawSegments(B,C C,D B,B' C,C' D,D')
\tkzDrawSegments[dashed](A,B A,D  A,A')
\draw[->](A')--(-1.5,-1.5)node[left]{$X$};
\draw[->](B)--(3,0)node[right]{$Y$};
\draw[->](D)--(0,3.5)node[right]{$Z$};
\draw[->, dashed](A)--(C');
\draw[->, very thick](0,0)--(0,1)node[left]{$\eZ$};
\draw[->, very thick](0,0)--node[above]{$\eY$}(1,0);
\draw[->, very thick](0,0)--(-.5,-.5)node[above]{$\eX$};
\node at (0,0)[below right]{$O$};
\node at (C')[right]{$P$};
\end{tikzpicture}
    \caption{}
\end{figure}

已知$A(x_1,y_1,z_1)$, $B(x_2,y_2,z_2)$, 则:
\[\begin{split}
   \Vec{AB}&=\Vec{OB}-\Vec{OA}\\
&=x_2\eX+y_2\eY+z_2\eZ-(x_1\eX+y_1\eY+z_1\eZ)\\
&=(x_2-x_1)\eX+(y_2-y_1)\eY+(z_2-z_1)\eZ\\
&=(x_2-x_1, y_2-y_1, z_2-z_1)
\end{split}\]
这就是说\textbf{一个向量的坐标,等于表示它的有向线段终点的坐
标减去起点的坐标}。例如,已知$A(2,-1,5)$、$B(3,
2,-7)$, 则
\[\Vec{AB}=[3-2,\; 2-(-1),\; -7-5]=(1,3,-12)\]

\begin{ex}
\begin{enumerate}
    \item 问在$OXYZ$中,哪个坐标平面与$X$轴垂直,哪个坐标
    平面与$Y$轴垂直,哪个坐标平面与$Z$轴垂直?
    \item 写出点$P(2,4,3)$在$OXYZ$的三个坐标平面上投
    影点的坐标。
    \item 求点$P(3,5,4)$关于坐标平面$OXY$的对称点的坐
    标。
    \item 点$P$在$OXYZ$中的坐标平面$OXY$上,若$P$点在平面
    直角坐标系$OXY$中的坐标是$(2,3)$, 求它在
    $OXYZ$中的坐标。
    \item 写出基向量$\eX,\eY,\eZ$的坐标。
    \item 已知$\vec{a}=12$, $\langle\eX ,\vec{a}\rangle=30^{\circ}$, $\langle\eY ,\vec{a}\rangle=45^{\circ}$, $\langle\eZ ,\vec{a}\rangle=60^{\circ}$,
求$\vec{a}$的坐标。
    \item 已知$P(-3,2,4)$, $Q(5,7,-2)$, 求$\Vec{PQ}$与$\Vec{QP}$的坐标。
    \item 已知$A(2,-1,5)$, $B(3,2,-1)$用基向量$\eX,\eY,\eZ$表示向量$\Vec{AB}$.
\end{enumerate}
\end{ex}

\subsection{向量的坐标运算}

\begin{blk}{定理}
     如果$\vec{a}=(a_x,a_y,a_z)$, $\vec{b}=(b_x, b_y,b_z)$, 
$\vec{c}=(c_x,c_y,c_z)$, 那么
\[\begin{split}
    \vec{a}\pm \vec{b}&=(a_x,a_y,a_z)\pm (b_x,b_y,b_z)
=(a_x\pm b_x,a_y\pm b_y,a_z\pm b_z)\\
\lambda\vec{a}&=\lambda(a_x, a_y,a_z)=(\lambda a_x,\lambda 
a_y,\lambda a_z)\\
\vec{a}\cdot \vec{b}&=(a_x, a_y, a_z)\cdot (b_x,b_y,b_z)
=a_xb_x+a_yb_y+a_zb_z\\
\end{split}\]
\[\vec{a}\x \vec{b}=\begin{vmatrix}
  \eX&\eY&\eZ\\
  a_x&a_y&a_z\\
  b_x&b_y&b_z  
\end{vmatrix},\qquad \left(\vec{a},\vec{b},\vec{c}\right)=\begin{vmatrix}
    a_x&a_y&a_z\\
    b_x&b_y&b_z\\  
    c_x&c_y&c_z
\end{vmatrix}\]
\end{blk}


证明留给同学作为练习。

下面我们研究如何用向量的坐标来表示向量垂直、平行
与共面的条件。

已知$\vec{a}\parallel \vec{b}\quad (\vec{b}\ne 0)$的充要条件是存在一实数$\lambda$,使
$$\vec{a}=\lambda\vec{b}$$
如果
$\vec{a}=(a_x,a_y,a_z)$, $\vec{b}=(b_x,b_y,b_z)$, 那么上面
条件用坐标表示,即为
\begin{equation}
    a_x=\lambda b_x,\qquad  a_y=\lambda b_y,\qquad  a_z=\lambda b_z
\end{equation}
或
\begin{equation}
    a_x:b_x=a_y:b_y=a_z:b_z
\end{equation}
这就是说\textbf{两个向量平行的充要条件是它们的坐标成比例}。

已知$\vec{a}\bot \vec{b}\quad \Longleftrightarrow \quad \vec{a}\cdot \vec{b}=0$
用坐标表示,即为
\begin{equation}
    \vec{a}\bot \vec{b}\quad \Longleftrightarrow \quad a_xb_x+a_yb_y+a_zb_z=0
\end{equation}




























































































































































\begin{ex}
\begin{enumerate}
    \item 求通过点$P_0(-1,2,-3)$且平行于向量$\vec{s}=(2,3,-5)$
    的直线方程。
    \item 求通过$P_0(2,3,1)$, $P_1(-1,-2,3)$的直线方程。
    \item 求通过点$(2,3,1)$且和$X$轴平行的直线方程。
    \item 求过点$(2,-3,7)$, 其方向向量为$(2,0,3)$的直
    线方程。
    \item 求直线$2x-6=4-y=2-5$的方向向量。
    \item 求平行于两平面$x-2y+5z+2=0$和$3x+y-z
    +5=0$的交线,且通过原点的直线方程。
\end{enumerate}
\end{ex}


\subsection{球面方程}
空间一动点$P(x,y,z)$在以$A(a,b,c)$为球
心,$R$为半径的球面上的充要条件是
\[|\Vec{OP}-\Vec{OA}|=R\]
或
\[(\Vec{OP}-\Vec{OA})\cdot (\Vec{OP}-\Vec{OA})=R^2\]
换用坐标表示,条件可写为
\begin{equation}
    (x-a)^2+(y-b)^2+(z-c)^2=R^2
\end{equation}

\begin{figure}[htp]
    \centering
\begin{tikzpicture}[>=latex]
\draw[<->](0,4)node[right]{$Z$}--(0,0)node[below right]{$O$}--(3.5,0)node[right]{$Y$};
\draw[->](0,0)--(-1,-1)node[right]{$X$};
\tkzDefPoints{1.5/1.75/A, 0/0/O, 1/2.62/P}

\draw[thick](A) circle (1);
\draw[dashed](A) ellipse[x radius=1, y radius=.4];
\draw[thick](2.5,1.75) arc [x radius=1, y radius=.4,start angle =0, end angle =-180];
\tkzDrawSegments[dashed, ->](O,P A,P O,A)
\tkzLabelPoints[right](A)
\tkzLabelPoints[above](P)
\end{tikzpicture}
    \caption{}
\end{figure}


方程(8.10)就是以$A(a,b,c)$为球心,以$R$为半径的\textbf{球面
方程}。当$A$点在原点,球面方程变为
\[    x^2+y^2+z^2=R^2\]



\begin{ex}
\begin{enumerate}
    \item 求以$A(1,2,-2)$为球心,3为半径的球面方程。
    \item 求球心在原点,半径等于5的球面方程。
    \item 设一动点$Q$在以$A(0,4,0)$为球心,2为半径的球面
    上变动,求$\overline{OQ}$中点的轨迹。
\end{enumerate}
\end{ex}

\subsection*{习题8.2}
\begin{enumerate}
    \item 求满足以下条件的平面方程。
\begin{enumerate}
\item 通过点$P_0(5,3,4)$且垂直于向量$\vec{n}=(1,1,1)$;
\item 通过坐标原点且垂直于$\vec{n}=\left(-\frac{1}{3},\frac{2}{3},-\frac{2}{3}\right)$;
\item 垂直于$\vec{n}=\left(\frac{1}{2},\frac{\sqrt{3}}{2},0\right)$且与原点的距离等于5.
\end{enumerate}

    \item 说出如下方程表示的平面的几何特征。
\begin{multicols}{3}
\begin{enumerate}
    \item $x=2$ \item $x=y$ \item $x+y+z=1$
\end{enumerate}
\end{multicols}

    \item 求证通过三点$A(a,0,0)$, $B(0,b,0)$, $C(0,
    0,c)$的平面方程为
\[\frac{x}{a}+\frac{y}{b}+\frac{z}{c}=1\] 

\item 如图,试写出长方体$ABC
D-A'B'C'D'$的各侧面,底面的平面方程以及各
棱所在的直线方程。已知
$\overline{AB}=a$, $\overline{AD}=b$, 
$\overline{AA'}=c$.

\begin{figure}[htp]
    \centering
\begin{tikzpicture}[>=latex, scale=1.3]
    \tkzDefPoints{0/0/A, -.5/-.5/B, 1/-.5/C, 1.5/0/D, 0/2/A'}
    \tkzDefPointsBy[translation= from A to A'](B,C,D){B',C',D'}
    \tkzDrawPolygon[thick](A',B',C',D')
    \tkzDrawPolygon[dashed](A,B,C,D)
    \tkzDrawSegments[thick](B,C C,D B,B' C,C' D,D')
    \tkzDrawSegments[dashed](A,B A,D  A,A')
    \draw[->](B)--(-1,-1)node[left]{$X$};
    \draw[->](D)--(2.5,0)node[right]{$Y$};
    \draw[->](A')--(0,3)node[right]{$Z$};
    \node at (0,0)[left]{$O$};
\tkzLabelPoints[below](B,C,A,D)
\tkzLabelPoints[left](B',A')
\tkzLabelPoints[right](C',D')
\end{tikzpicture}
    \caption*{第4题}
\end{figure}


\item 分别求两点$P_1(3,9,1)$, $P_2(4,1,5)$到平面$S:\;
x-2y+2z-3=0$的距离。
\item 求两条直线
\[\begin{split}
    \ell_1:&\quad \frac{x-1}{3}=\frac{y+2}{6}=\frac{z-5}{2}\\
    \ell_2:&\quad \frac{x}{2}=\frac{y-3}{9}=\frac{z+1}{6}\\
\end{split}\]
的夹角。
\item 求通过点$(1,-1,2)$并与已知平面:$x+y+z=1$
垂直的直线方程。并求这条直线与平面交点的坐标。
\item 在直线$\ell:\; x=1+2t,\; y=8+t,\; z=8+3t$上求
一点使它和原点的距离等于35.
\item 求满足下列条件的球面方程。
\begin{multicols}{2}
    \begin{enumerate}
    \item 球心在$(-2,3,-6)$, 半径是7
    \item 球心在$(4,0,0)$, 半径是2
    \item 球心在$(0,-4,3)$, 半径是5
    \item 球心在$(0,-5,0)$, 半径是2
    \item 球心在$\left(\frac{2}{3},-\frac{1}{3},0\right)$,
半径是1
\end{enumerate}
\end{multicols}

\item 求球面:$x^2+y^2+z^2+4x-6y-2z+5=0$的球心和
半径。
\end{enumerate}

\section*{复习题八}
\begin{enumerate}
\item 已知点$P(3,-1,2)$和$M(a,b,c)$, 求$P$、$M$两点
分别关于坐标平面、坐标轴以及原点的对称点的坐标。
\item 求点$P(2,5,6)$到坐标原点以及三条坐标轴的距离。
\item 已知$\Vec{OA}=(6,2,9)$,求$\Vec{OA}$与三个坐标平面的夹角.
\item 已知$A(-2,1,3)$, $B(0,-1,2)$,求与$A$、$B$两点
距离相等点的轨迹方程。
\item 已知$A(a,0,0)$, $B(0,b,0)$, $C(0,0,c)$, 原
点到平面$(A,B,C)$的距离为$d$, 求证
\[\frac{1}{a^2}+\frac{1}{b^2}+\frac{1}{c^2}=\frac{1}{d^2}\]
\item 在$Z$轴上求一点,使得到$A(-4,1,7)$, $B(3,5,-2)$两点的距离相等。
\item 已知四面体$ABCD$,且$A(x_1,y_1,z_1)$, $B(x_2,
y_2,z_2)$, $C(x_3,y_3, z_3)$, $D(x_4, y_4,z_4)$, 
求它的重心的坐标。
\item 已知$\vec{a}=(a_1,a_2,a_3)$, $\vec{b}=(b_1,b_2,b_3)$, 求证:以$\vec{a}$, $\vec{b}$为邻边的平行四边形面积
\[S=\sqrt{\begin{vmatrix}
    a_2&a_3\\b_2&b_3
\end{vmatrix}^2+\begin{vmatrix}
    a_3&a_1\\b_3&b_1
\end{vmatrix}^2+\begin{vmatrix}
    a_1&a_2\\b_1&b_2
\end{vmatrix}^2}\]

\item 已知点$P(1,3,5)$和$\vec{a}(-2,1,1)$. 求通过$P$点
具有方向$\vec{a}$的直线与平面$2x+3y-z=1$的交点。

\item 求通过如下三点的平面方程。
\begin{enumerate}
    \item $(2,1,1),\qquad (3,-1,1),\qquad (4,1,-1)$
    \item $(-2,3,-1),\qquad (2,2,3),\qquad (-4,-1,1)$
    \item $(-5,-1,-2),\qquad (1,2,-1),\qquad (3,-1,2)$
\end{enumerate}

\item 求下列通过已知点$P$且以$\vec{a}$为方向向量的直线方程。
\begin{multicols}{2}
    \begin{enumerate}
    \item $P(2,1,3),\quad \vec{a}=(1,1,-2)$
    \item $P(-5,3,4),\quad \vec{a}=(-2,2,1)$
    \item $P(4,-3,2),\quad \vec{a}=(5,0,3)$
    \item $P(0,0,0),\quad \vec{a}=(2,-3,5)$
    \item $P(a,b,c),\quad \vec{a}=(\ell,m,n)$
\end{enumerate}
\end{multicols}

\end{enumerate}

